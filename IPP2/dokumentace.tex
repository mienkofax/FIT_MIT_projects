\documentclass[10pt,a4paper,final]{article}
% cestina a fonty

\usepackage[utf8]{inputenc}
\usepackage[T1]{fontenc}
\usepackage{lmodern}
\usepackage{textcomp}
\usepackage{times}
% odsazeni prvniho radku
\usepackage{indentfirst}
% balicky pro odkazy
\usepackage[bookmarksopen,colorlinks,plainpages=false,urlcolor=blue,
unicode,linkcolor=black]{hyperref}
\usepackage{url}
% obrazky
\usepackage[dvipdf]{graphicx}
% velikost stranky
\usepackage[top=3.5cm, left=2.5cm, text={17cm, 24cm}, ignorefoot]{geometry}

\begin{document}

%%%%%%%%%%%%%%%%%%%%%%%%%%%%%%%%%%%%%%%%%%%%%%%%%%%%%%%%%%%%%%%%%%%%%%%%%%%%%%%%

  % nastaveni cislovani
  \pagestyle{plain}
  \pagenumbering{arabic}
  \setcounter{page}{1}
  
  % nastaveni mezery mezi odstavci a odsazeni prvniho radku
  \setlength{\parindent}{1cm}
  \setlength{\parskip}{0.5cm plus4mm minus3mm}
  
  \noindent
  Dokumentace úlohy CSV: CSV2XML v Python 3 do IPP 2015/2016 \\
  Jméno a příjmení: Peter Tisovčík \\
  Login: xtisov00 \\
  


%%%%%%%%%%%%%%%%%%%%%%%%%%%%%%%%%%%%%%%%%%%%%%%%%%%%%%%%%%%%%%%%%%%%%%%%%%%%%%%%
  \section{Úvod} \label{uvod}
%%%%%%%%%%%%%%%%%%%%%%%%%%%%%%%%%%%%%%%%%%%%%%%%%%%%%%%%%%%%%%%%%%%%%%%%%%%%%%%%

Tato dokumentace popisuje implementaci skriptu v~jazyce Python 3. Úkolem vytvořeného skriptu je převod CSV souboru do odpovídajícího XML výstupu na základě zadaných parametrů. V dokumentaci se nachází popis implementace zpracování parametrů programu, jejich přípustné kombinace a následného generování výstupního souboru v XML formátu.

%%%%%%%%%%%%%%%%%%%%%%%%%%%%%%%%%%%%%%%%%%%%%%%%%%%%%%%%%%%%%%%%%%%%%%%%%%%%%%%%
  \section{Zpracování parametrů} \label{zpracovani-parametru}
%%%%%%%%%%%%%%%%%%%%%%%%%%%%%%%%%%%%%%%%%%%%%%%%%%%%%%%%%%%%%%%%%%%%%%%%%%%%%%%%

Ke zpracování vstupních parametrů slouží třída \texttt{ArgumentParser()}. Samotná třída však nestačí k ošetření všech možností, které ze zadání vyplývají. Tyto nedostatky řeší metoda \texttt{parseArg()}. Tato metoda ošetřuje situace, kdy jsou zadány nesprávné parametry, duplicitní parametry, nesprávné hodnoty parametrů jako například neexistující vstupní soubor a zadání nevalidní hodnoty. Také ošetřuje různé přípustné a nepřípustné kombinace parametrů dle zadání. Menším nedostatkem třídy \texttt{ArgumentParser()} bylo automatické vytvoření výjimky v případě zadání nesprávných parametrů. Tento problém by vyřešen zděděním dané třídy a~přepsáním metody \texttt{error()} tak, aby vracela chybový kód dle zadání. V těle této metody se také generuje nápověda k~programu. 
	
%%%%%%%%%%%%%%%%%%%%%%%%%%%%%%%%%%%%%%%%%%%%%%%%%%%%%%%%%%%%%%%%%%%%%%%%%%%%%%%%
  \section{Načtení CSV souboru} \label{nacitanie-suboru}
%%%%%%%%%%%%%%%%%%%%%%%%%%%%%%%%%%%%%%%%%%%%%%%%%%%%%%%%%%%%%%%%%%%%%%%%%%%%%%%%

Pro usnadnění zpracování vstupního CSV souboru slouží metoda \texttt{reader()} z~modulu \texttt{csv}, která načítá soubor a~vytvoří z něj CSV strukturu, skrz kterou lze iterovat. Při načítání CSV souboru je možné pomocí přepínače \texttt{-s} specifikovat oddělovač CSV záznamů. 
%\textit{TODO: a ako sa oznašuje reťazec, ktorý sa má vypísať, v našom prípade sú to úvodzovky.} 
Také bylo nutné vyřešit problém s~importováním modulu \texttt{csv}, protože se jmenoval stejně jako skript, do kterého se importoval.

%%%%%%%%%%%%%%%%%%%%%%%%%%%%%%%%%%%%%%%%%%%%%%%%%%%%%%%%%%%%%%%%%%%%%%%%%%%%%%%%
  \section{Generování výstupního XML} \label{generovanie-xml}
%%%%%%%%%%%%%%%%%%%%%%%%%%%%%%%%%%%%%%%%%%%%%%%%%%%%%%%%%%%%%%%%%%%%%%%%%%%%%%%%

Pro generování XML výstupu byl použit modul \texttt{ElementTree}. Objekt této třídy reprezentuje strukturu výstupního souboru. Při zpracování jednotlivých řádků a~sloupců se vstup zpracovává podle zadaných parametrů. Pokud je zadán přepínač \texttt{-h}, potom je první řádek vstupního CSV souboru použit pro odvození názvů sloupců výstupního XML souboru. V~opačném případě se využije předdefinovaný řetězec, které je uložený v~seznamu. Pomocí něj se kontroluje počet sloupců.  

V~případě načítání hlavičky z CSV souboru je nutné ošetřit případy, kdy načtený řetězec obsahuje znaky, které vedou na vznik neplatného XML elementu. Takovými znaky jako například bílé znaky, číslice vyskytující se na začátku řetězce a~další znaky znakové sady UTF-8. Kontrolu validity XML elementu provádějí metody \texttt{validElement()} a~\texttt{replaceElement()}. Tyto metody ošetřují, zda se znak daného řetězce nachází v~povoleném rozsahu dle standardu. Rozsahy se liší podle toho, zda se jedná o~ první znak či o~ostatní znaky v~řetězci. 

Také je potřeba kontrolovat obsahy jednotlivých elementů a problematické znaky převést na odpovídající XML entity. Dále je potřeba udržovat informaci o~počtu údajů (sloupců) na řádek, zda je na daném řádku dostatek sloupců, zda některý sloupec nepřebývá. Pokud byl zadán odpovídající přepínač, nedojde k chybě, ale doplní se chybějící sloupec, v~opačném případě bude přebytečný sloupec ignorován. Údaje se postupně předávají do XML objektu a~po skončení načítání vstupního souboru se výsledek předá metodě \texttt{generateXML()}, která vytvoří výstupní XML soubor. Při ukládání XML souboru se nejprve vygeneruje XML jako řetězec a~uloží se do proměnné. V~procesu generování se volí, zda se má vypsat i~ XML hlavička a~zda se mají XML elementy odsazovat.     
   

%%%%%%%%%%%%%%%%%%%%%%%%%%%%%%%%%%%%%%%%%%%%%%%%%%%%%%%%%%%%%%%%%%%%%%%%%%%%%%%%
  \section{Testování} \label{testovanie}
%%%%%%%%%%%%%%%%%%%%%%%%%%%%%%%%%%%%%%%%%%%%%%%%%%%%%%%%%%%%%%%%%%%%%%%%%%%%%%%%

Důležitou součástí celého projektu bylo jeho testování. K tomuto účelu byl vytvořen bash skript, který nejdříve využíval poskytnuté testy a~následně byly doplněny i testy vlastní. Testy porovnávaly výstupní soubory a~chybové kódy v~případě nesprávného zpracování s~předpokládanými výstupy. Testování probíhalo automaticky, pro porovnání výsledných XML souborů byl využit doporučený nástroj JExamXML. Nástroj JExamXML nepodporuje XML strukturu, která neobsahuje kořenový element. Proto musel každý test obsahovat přepínač, pomocí kterého bylo možné chybějící kořenový element nastavit. 
    
\end{document}
