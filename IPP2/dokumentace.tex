\documentclass[10pt,a4paper,final]{article}
% cestina a fonty
\usepackage[czech]{babel}
\usepackage[utf8]{inputenc}
\usepackage[T1]{fontenc}
\usepackage{lmodern}
\usepackage{textcomp}
\usepackage{times}
% odsazeni prvniho radku
\usepackage{indentfirst}
% balicky pro odkazy
\usepackage[bookmarksopen,colorlinks,plainpages=false,urlcolor=blue,
unicode,linkcolor=black]{hyperref}
\usepackage{url}
% obrazky
\usepackage[dvipdf]{graphicx}
% velikost stranky
\usepackage[top=3.5cm, left=2.5cm, text={17cm, 24cm}, ignorefoot]{geometry}

\begin{document}

%%%%%%%%%%%%%%%%%%%%%%%%%%%%%%%%%%%%%%%%%%%%%%%%%%%%%%%%%%%%%%%%%%%%%%%%%%%%%%%%

  % nastaveni cislovani
  \pagestyle{plain}
  \pagenumbering{arabic}
  \setcounter{page}{1}
  
  % nastaveni mezery mezi odstavci a odsazeni prvniho radku
  \setlength{\parindent}{1cm}
  \setlength{\parskip}{0.5cm plus4mm minus3mm}
  
  \noindent
  Dokumentace úlohy CSV: CSV2XML v Python3 do IPP 2015/2016 \\
  Jméno a příjmení: Peter Tisovčík \\
  Login: xtisov00 \\
  


%%%%%%%%%%%%%%%%%%%%%%%%%%%%%%%%%%%%%%%%%%%%%%%%%%%%%%%%%%%%%%%%%%%%%%%%%%%%%%%%
  \section{Úvod} \label{uvod}
%%%%%%%%%%%%%%%%%%%%%%%%%%%%%%%%%%%%%%%%%%%%%%%%%%%%%%%%%%%%%%%%%%%%%%%%%%%%%%%%

Tato dokumentace popisuje implementaci skriptu v jazyce Python3. Úkolem vytvořeného skriptu je prevod CSV vstupu do jeho odpovedajúceho XML výstupu na základe zadaných parametrov. V dokumentaci se nachází popis implementace zpracování parametrů programu, ich prípustnej kombinácií a následného generování výstupního souboru v XML formátu, ktoré sa dá presne našpecifikovať na základe zadaných parametrov.

%%%%%%%%%%%%%%%%%%%%%%%%%%%%%%%%%%%%%%%%%%%%%%%%%%%%%%%%%%%%%%%%%%%%%%%%%%%%%%%%
  \section{Zpracování parametrů} \label{zpracovani-parametru}
%%%%%%%%%%%%%%%%%%%%%%%%%%%%%%%%%%%%%%%%%%%%%%%%%%%%%%%%%%%%%%%%%%%%%%%%%%%%%%%%

Ke zpracování vstupních parametrů slouží trieda \texttt{ArgumentParser()}, která načte parametry na základě zadaných přepínačů. Samotná trieda však nestačí k ošetření všech možností, které ze zadání vyplývají. Tyto nedostatky řeší metóda \texttt{parseArg()}. Metóda ošetřuje situace, kdy jsou zadány nesprávné argumenty, duplicitné argumenty, nesprávné parametry argumentů jako například neexistující vstupní soubor a zadání nevalidnej hodnoty. Taktiež ošetruje rôzne prípustné a neprípustne kombinácie prepínačov podľa zadania. Menším nedostatkom triedy \texttt{ArgumentParser()} bolo automatické vytvorenie výnimky pri nesprávnych argumentoch alebo nesprávnych hodnotách argumentov. V mojom prípade som to vyriešil zdedením triedy a prepísaním metódy \texttt{error} aby vracala chybu podľa zadania. V tele tejto metódy sa taktiež generuje nápoveda programu.
	

%%%%%%%%%%%%%%%%%%%%%%%%%%%%%%%%%%%%%%%%%%%%%%%%%%%%%%%%%%%%%%%%%%%%%%%%%%%%%%%%
  \section{Načtení CSV súboru} \label{nacitanie-suboru}
%%%%%%%%%%%%%%%%%%%%%%%%%%%%%%%%%%%%%%%%%%%%%%%%%%%%%%%%%%%%%%%%%%%%%%%%%%%%%%%%
Pro usnadnění zpracování vstupního CSV souboru slouží metóda \texttt{reader()} z modulu csv, ktorá načíta súbor a vytvori z neho csv štruktúru, cez ktorú sa dá iterovať. Pri načítavaní CSV súboru je možné špecifikovať aký oddeľovač má byť použitý na rozdelenie stĺpcov a ako sa oznašuje reťazec, ktorý sa má vypísať, v našom prípade sú to úvodzovky. Pri načítaní tohto modulu bolo potrebné vyriešit problém s jej importovaním, pretože modul sa volal rovnako ako script, do ktorého sa imporoval.


%%%%%%%%%%%%%%%%%%%%%%%%%%%%%%%%%%%%%%%%%%%%%%%%%%%%%%%%%%%%%%%%%%%%%%%%%%%%%%%%
  \section{Generování výstupního XML} \label{generovanie-xml}
%%%%%%%%%%%%%%%%%%%%%%%%%%%%%%%%%%%%%%%%%%%%%%%%%%%%%%%%%%%%%%%%%%%%%%%%%%%%%%%%

Pro generování XML výstupu byl použitý modul \texttt{ElementTree}. Objekt z této třídy reprezentoval strukturu výstupného súboru. Pri prechode jednotlivými riadkami a stĺpcami sa spracovával vstup podľa zadaných parametrov. Ako hlavička sa môže použiť prvý riadok vstupného súboru alebo preddefinovaný reťazec, ktorý je uložený v liste a následne sa pomocou neho kontroluje počet stĺpcov. 

V prípade, že sa jedná o načítanie hlavičky z CSV súboru je potrebné ošetriť situácie, kedy môže obsahovať reťazec nepovolené znaky XML elementu ako rôzne biele znaky alebo čísla na prvom mieste a iné znaky znakovej sady UTF-8, túto kontrolu rieši metóda \texttt{validElement} a  \texttt{replaceElement}. Tieto metódy kontrolujú, či sa nachádza znak v povolenom rozsahu podľa štandardu. Rozlišné pravidlá platia pre prvý a pre ostatné znaky. Obsah jednotlivých elementov je potreba kontrolovať , či neobsahuje náhodou znaky, ktoré treba previesť na odpovedajúce XML entity. Ďalej je potrebné počítať počet údajov na riadok, či na danom riadku je dostatok stĺpcov alebo, či niektorý stĺpec nie je navyše a v prípade, že bol zadaný odpovedajúci prepínač sa nevypíše chyba ale dolpní sa chýbajúci stĺpec alebo prebytočný stĺpec bude ignorovaný. Údaje sa postupne pridávajú do XML objektu a po skončení načítavania vstupného súboru sa predá metóde \texttt{generateXML()}, ktorá uloží výstup do súboru. Pri ukladaní XML do súboru sa najpr vygeneruje XML ako reaťazec a uloží sa do premennej. Pri generovaní sa volí, či sa má vypísať aj hlavička XML a či sa majú odsadiť elementy. 
   

%%%%%%%%%%%%%%%%%%%%%%%%%%%%%%%%%%%%%%%%%%%%%%%%%%%%%%%%%%%%%%%%%%%%%%%%%%%%%%%%
  \section{Testovanie} \label{testovanie}
%%%%%%%%%%%%%%%%%%%%%%%%%%%%%%%%%%%%%%%%%%%%%%%%%%%%%%%%%%%%%%%%%%%%%%%%%%%%%%%%

Dôležitou súčasťou celého projektu bolo testovanie. Na testovanie som si vytvoril vlastný bash script, ktorý využíval zo začiatku poskytnuté testy a neskôr som boli doplnené o vlastné testy. Testy porovnávali výstupné súbory a chybové kódy v prípade nesprávneho spracovania a porovnávali sa s predpokladaným výstupom. Testovanie prebiehalo automaticky a na porovnávanie výsledných xml súborov bol použitý odporúčaný nástroj JExamXML. Nástroj JExamXML nepodporuje XML štruktúru, ktorá neobsahuje root element, preto každý test musel obsahovať prepínač, ktorým sa nastavil root element.

    
\end{document}
